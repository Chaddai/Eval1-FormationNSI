% Options for packages loaded elsewhere
\PassOptionsToPackage{unicode}{hyperref}
\PassOptionsToPackage{hyphens}{url}
%
\documentclass[french]{scrartcl}
\usepackage{lmodern}
\usepackage{amssymb,amsmath}
\usepackage{ifxetex,ifluatex}
\ifnum 0\ifxetex 1\fi\ifluatex 1\fi=0 % if pdftex
  \usepackage[T1]{fontenc}
  \usepackage[utf8]{inputenc}
  \usepackage{textcomp} % provide euro and other symbols
\else % if luatex or xetex
  \usepackage{unicode-math}
  \defaultfontfeatures{Scale=MatchLowercase}
  \defaultfontfeatures[\rmfamily]{Ligatures=TeX,Scale=1}
\fi
% Use upquote if available, for straight quotes in verbatim environments
\IfFileExists{upquote.sty}{\usepackage{upquote}}{}
\IfFileExists{microtype.sty}{% use microtype if available
  \usepackage[]{microtype}
  \UseMicrotypeSet[protrusion]{basicmath} % disable protrusion for tt fonts
}{}
\makeatletter
\@ifundefined{KOMAClassName}{% if non-KOMA class
  \IfFileExists{parskip.sty}{%
    \usepackage{parskip}
  }{% else
    \setlength{\parindent}{0pt}
    \setlength{\parskip}{6pt plus 2pt minus 1pt}}
}{% if KOMA class
  \KOMAoptions{parskip=half}}
\makeatother
\usepackage{xcolor}
\IfFileExists{xurl.sty}{\usepackage{xurl}}{} % add URL line breaks if available
\IfFileExists{bookmark.sty}{\usepackage{bookmark}}{\usepackage{hyperref}}
\hypersetup{
    colorlinks=true,
    linkcolor=blue,
    filecolor=magenta,      
    urlcolor=cyan,
}
\urlstyle{same} % disable monospaced font for URLs
\setlength{\emergencystretch}{3em} % prevent overfull lines
\providecommand{\tightlist}{%
  \setlength{\itemsep}{0pt}\setlength{\parskip}{0pt}}
\setcounter{secnumdepth}{-\maxdimen} % remove section numbering

\title{Correction de l'évaluation 1}
\author{}
\date{}

\usepackage{ccicons}
\usepackage{fancyhdr}

\pagestyle{fancy}
\fancyhf{}
\lfoot{\thepage}
\rfoot{\begin{minipage}{0.5\textwidth}
	\footnotesize Ce \href{https://github.com/Chaddai/Eval1-FormationNSI/tree/elementary}{travail} de Chaddaï Fouché est mise à disposition
selon les termes de \href{http://creativecommons.org/licenses/by-nc/4.0/}{la licence Creative Commons Attribution -- Pas d'utilisation commerciale} \ccbync
\end{minipage}}
\renewcommand{\headrulewidth}{0pt}

\begin{document}

\maketitle
\thispagestyle{fancy}

\hypertarget{partie-1}{%
\subsection{Partie 1}\label{partie-1}}

\begin{enumerate}
\def\labelenumi{\arabic{enumi}.}
\item
  On construit un tableau des ``questions'', chaque question étant
  constitué de

  \begin{itemize}
  \tightlist
  \item
    la question posée,
  \item
    la réponse attendue
  \item
    et le nombre de points attribués.
  \end{itemize}

  Ainsi chaque question sera représentée par un triplet de deux chaînes
  de caractères et d'un entier (str,str,int). Ce tableau est bien
  homogène et sera modifié par la suite (pour permettre le mélange par
  l'algorithme de Fisher-Yates) tandis que les questions sont immuables
  et hétérogènes (dû à l'inclusion des points) et doivent donc être
  représentées par un p-uplet.

  Les opérations à effectuer (mélange, sélection d'une sous-liste,
  parcours dans l'ordre) sont particulièrement adaptées à un tableau.
\item
  Voir le code python de la \href{partie1.py}{partie 1} ainsi que les
  modules \href{tableau.py}{tableau}, \href{chaine.py}{chaine} et
  \href{eval1.py}{eval1} qui contiennent des fonctions utilitaires
  créées précédemment dans l'année (ou rajoutées à l'occasion comme la
  fonction \texttt{demanderEntierEntre()}).
\end{enumerate}

\begin{center}\rule{0.5\linewidth}{0.5pt}\end{center}

\hypertarget{partie-2}{%
\subsection{Partie 2}\label{partie-2}}

\begin{enumerate}
\def\labelenumi{\arabic{enumi}.}
\setcounter{enumi}{2}
\item
  On peut réaliser ce choix à l'aide des tableaux utilisés dans la
  partie 1 mais cela est assez peu pratique, impliquant un parcours
  complet de tableau pour retrouver les questions correspondant à chaque
  nombre de points.

  Il est plus pragmatique et lisible de créer un dictionnaire lors de la
  lecture du fichier dont les clés seront le nombre de points par
  question possibles et la valeur associée est le tableau des questions
  rapportant ce nombre de point. Voir la \href{partie2.py}{partie 2}
  pour la réalisation.
\item
  Notez que cette partie a été rédigée en supposant que l'utilisateur
  veut toujours pouvoir choisir le nombre de questions par points.

  S'il souhaite pouvoir utiliser le mode de sélection de la partie 1 de
  temps en temps, il est préférable de garder une liste mère de toutes
  les questions également et de lui proposer l'alternative, voir la
  \href{partie2bis.py}{partie 2bis} pour une implémentation (notez que
  les triplets représentant les questions sont partagés en mémoire entre
  la liste et le dictionnaire et que l'immuabilité des tuples en Python
  (et des str) garantit ici l'absence de bugs dus à ce partage).
\end{enumerate}

\begin{center}\rule{0.5\linewidth}{0.5pt}\end{center}

\hypertarget{partie-3}{%
\subsection{Partie 3}\label{partie-3}}

\begin{enumerate}
\def\labelenumi{\arabic{enumi}.}
\setcounter{enumi}{4}
\item
  On aura besoin d'un dictionnaire \texttt{sansAccent} dont les clés
  seront les lettres accentuées et les valeurs la version non-accentuée
  de la clé.

  On utilisera également un dictionnaire \texttt{soundexValeurs} pour
  sélectionner quel traitement effectuer pour chaque lettre lors de
  l'étape finale, avec comme clés les lettres majuscules et comme
  valeurs leur valeur numérique selon soundex, ainsi que 0 pour celles à
  supprimer. On reconstruira ainsi, caractère par caractère la version
  Soundex de la chaine (qu'on suppose constituée de majuscules sans
  accent à cette étape).
\item
  On notera que cette suppression des accents (et cédille) peut aussi
  être accompli par une utilisation adéquate d'Unicode en mettant
  d'abord sa chaîne en forme normale KD (qui décompose un caractère
  composé en la lettre de base et un caractère dit ``combinant''
  représentant l'accent) puis en supprimant les caractères
  ``combinants'' (accents, cédilles, etc).

  D'une façon plus basique, on peut remplacer chaque caractère par sa
  version non-accentuée si ce caractère est une clé de
  \texttt{sansAccent} (utilisation de l'opérateur \texttt{in} sur un
  dictionnaire) ou en la rajoutant à l'identique sinon. Voir
  \texttt{retirerAccent()} dans le module \href{chaine.py}{chaine}.
\item
  La fonction \texttt{enMajuscule()} n'a à se préoccuper que des lettres
  minuscules sans accent, il suffit donc de remplacer chacune de ces
  lettres par sa version majuscule en ajoutant la différence entre les
  codes ASCII du `a' minuscule et du `A' majuscule à leur code ASCII.
  Voir le module \href{chaine.py}{chaine}.
\item
  La fonction \texttt{retireDoublon()} examine chaque caractère excepté
  le premier et ne recopie ce caractère au bout de la sortie que s'il
  diffère du précédent. Voir le module \href{chaine.py}{chaine}.
\item
  Pour \texttt{codeSoundex()}, on met de côté la première lettre puis
  l'on examine chaque lettre restante et selon sa valeur associée dans
  \texttt{soundexValeurs}, on passe (valeur 0) ou on la remplace par sa
  valeur (transformée en chaine par \texttt{str()}). Enfin on renvoie ma
  concaténation de la première lettre mise de côté, et des trois
  premiers caractères de la version traduite (à laquelle on a ajouté
  \texttt{\textquotesingle{}000\textquotesingle{}} pour garantir qu'elle
  contienne bien trois caractères au moins). Voir le module
  \href{eval1.py}{eval1} pour l'implémentation.
\end{enumerate}

On peut alors reprendre la partie 2 en comparant la version Soundex des
réponses données et attendues.

\hypertarget{bonus}{%
\subsection{Bonus}\label{bonus}}

Il y a néanmoins un petit problème si vous exécutez ce code : si vous
tombez sur la question « Quel est le nom de l'encodage qui utilise le
code ASCII ? », la réponse « utf-8 » n'étant pas constituée entièrement
de lettres sera mal traitée par votre fonction \texttt{codeSoundex()} :
si vous avez négligé de vérifier qu'un caractère est bien une clé de
votre dictionnaire avant de procéder, votre programme plantera tandis
que si vous avez passé ces caractères sans modifications ou sans les
recopier, la valeur renvoyée sera incorrecte (notez que le 8 dans utf-8
est important vu l'existence de utf-16 et utf-32).

Une solution propre à ce problème serait de découper le texte à traiter
en parties constituées entièrement de lettres auxquelles on applique
Soundex et d'autres symboles qu'on laisse passer à l'identique. C'est ce
qui est fait dans la fonction \texttt{soundexParMot()} du module
\href{eval1.py}{eval1} et utilisé dans la \href{partie3.py}{version
finale du programme}.

\end{document}
